%1
\newcommand{\IMageTable} {
\begin{table}[h!]
    \centering
    \caption{Упрощенная схема цифрового изображения как двумерного массива пикселей}
    \renewcommand{\arraystretch}{1.5}

    \begin{tabular}{|c|c|c|c|}
        \hline
        \cellcolor{gray!20}P(0,0) & \cellcolor{gray!10}P(0,1) & \cellcolor{gray!10}P(0,2) & \cellcolor{gray!20}P(0,3) \\
        \hline
        \cellcolor{gray!10}P(1,0) & \cellcolor{gray!20}P(1,1) & \cellcolor{gray!20}P(1,2) & \cellcolor{gray!10}P(1,3) \\
        \hline
        \cellcolor{gray!20}P(2,0) & \cellcolor{gray!10}P(2,1) & \cellcolor{gray!10}P(2,2) & \cellcolor{gray!20}P(2,3) \\
        \hline
        \cellcolor{gray!10}P(3,0) & \cellcolor{gray!20}P(3,1) & \cellcolor{gray!20}P(3,2) & \cellcolor{gray!10}P(3,3) \\
        \hline
    \end{tabular}
\end{table}
}



%2
\newcommand{\PixelTable} {
\begin{table}[h!]
    \centering
    \renewcommand{\arraystretch}{1.5} % увеличим высоту строк
    \begin{tabular}{|c|c|}
    \hline
    \textbf{RGB значения} & \textbf{Цвет} \\
    \hline
    RGB(255,0,0)   & {\color{red}\ding{108}} \\
    RGB(0,255,0)   & {\color{green}\ding{108}} \\
    RGB(0,0,255)   & {\color{blue}\ding{108}} \\
    \hline
    \end{tabular}
    \caption{Представление пикселей с помощью RGB-значений и их цветов}
\end{table}
}


\newcommand{\AddedBlocks} {
\begin{table}[h!]
    \centering
    \caption{Расширенный блок 8×8 с добавленными пикселами}
    \renewcommand{\arraystretch}{1.5}
    \setlength{\tabcolsep}{0pt} % убираем отступы в ячейках

    \begin{tabular}{|*{8}{p{0.9cm}|}}
        \hline
        \cellcolor{gray!20} & \cellcolor{gray!20} & \cellcolor{gray!20} & \cellcolor{gray!20} & \cellcolor{gray!20} & \cellcolor{gray!20} & \cellcolor{gray!20} & \cellcolor{blue!30} \\
        \hline
        \cellcolor{gray!20} & \cellcolor{gray!20} & \cellcolor{gray!20} & \cellcolor{gray!20} & \cellcolor{gray!20} & \cellcolor{gray!20} & \cellcolor{gray!20} & \cellcolor{blue!30} \\
        \hline
        \cellcolor{gray!20} & \cellcolor{gray!20} & \cellcolor{gray!20} & \cellcolor{gray!20} & \cellcolor{gray!20} & \cellcolor{gray!20} & \cellcolor{gray!20} & \cellcolor{blue!30} \\
        \hline
        \cellcolor{gray!20} & \cellcolor{gray!20} & \cellcolor{gray!20} & \cellcolor{gray!20} & \cellcolor{gray!20} & \cellcolor{gray!20} & \cellcolor{gray!20} & \cellcolor{blue!30} \\
        \hline
        \cellcolor{gray!20} & \cellcolor{gray!20} & \cellcolor{gray!20} & \cellcolor{gray!20} & \cellcolor{gray!20} & \cellcolor{gray!20} & \cellcolor{gray!20} & \cellcolor{blue!30} \\
        \hline
        \cellcolor{gray!20} & \cellcolor{gray!20} & \cellcolor{gray!20} & \cellcolor{gray!20} & \cellcolor{gray!20} & \cellcolor{gray!20} & \cellcolor{gray!20} & \cellcolor{blue!30} \\
        \hline
        \cellcolor{gray!20} & \cellcolor{gray!20} & \cellcolor{gray!20} & \cellcolor{gray!20} & \cellcolor{gray!20} & \cellcolor{gray!20} & \cellcolor{gray!20} & \cellcolor{blue!30} \\
        \hline
        \cellcolor{blue!30} & \cellcolor{blue!30} & \cellcolor{blue!30} & \cellcolor{blue!30} & \cellcolor{blue!30} & \cellcolor{blue!30} & \cellcolor{blue!30} & \cellcolor{blue!30} \\
        \hline
    \end{tabular}
\end{table}
}