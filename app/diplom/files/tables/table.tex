%1
\newcommand{\IMageTable} {
\begin{table}[h!]
    \centering
    \caption{Упрощенная схема цифрового изображения как двумерного массива пикселей}
    \renewcommand{\arraystretch}{1.5}

    \begin{tabular}{|c|c|c|c|}
        \hline
        \cellcolor{gray!20}P(0,0) & \cellcolor{gray!10}P(0,1) & \cellcolor{gray!10}P(0,2) & \cellcolor{gray!20}P(0,3) \\
        \hline
        \cellcolor{gray!10}P(1,0) & \cellcolor{gray!20}P(1,1) & \cellcolor{gray!20}P(1,2) & \cellcolor{gray!10}P(1,3) \\
        \hline
        \cellcolor{gray!20}P(2,0) & \cellcolor{gray!10}P(2,1) & \cellcolor{gray!10}P(2,2) & \cellcolor{gray!20}P(2,3) \\
        \hline
        \cellcolor{gray!10}P(3,0) & \cellcolor{gray!20}P(3,1) & \cellcolor{gray!20}P(3,2) & \cellcolor{gray!10}P(3,3) \\
        \hline
    \end{tabular}
\end{table}
}



%2
\newcommand{\PixelTable} {
\begin{table}[h!]
    \centering
    \renewcommand{\arraystretch}{1.5} % увеличим высоту строк
    \begin{tabular}{|c|c|}
    \hline
    \textbf{RGB значения} & \textbf{Цвет} \\
    \hline
    RGB(255,0,0)   & {\color{red}\ding{108}} \\
    RGB(0,255,0)   & {\color{green}\ding{108}} \\
    RGB(0,0,255)   & {\color{blue}\ding{108}} \\
    \hline
    \end{tabular}
    \caption{Представление пикселей с помощью RGB-значений и их цветов}
\end{table}
}


\newcommand{\AddedBlocks} {
\begin{table}[h!]
    \centering
    \caption{Расширенный блок 8×8 с добавленными пикселами}
    \renewcommand{\arraystretch}{1.5}
    \setlength{\tabcolsep}{0pt} % убираем отступы в ячейках

    \begin{tabular}{|*{8}{p{0.9cm}|}}
        \hline
        \cellcolor{gray!20} & \cellcolor{gray!20} & \cellcolor{gray!20} & \cellcolor{gray!20} & \cellcolor{gray!20} & \cellcolor{gray!20} & \cellcolor{gray!20} & \cellcolor{blue!30} \\
        \hline
        \cellcolor{gray!20} & \cellcolor{gray!20} & \cellcolor{gray!20} & \cellcolor{gray!20} & \cellcolor{gray!20} & \cellcolor{gray!20} & \cellcolor{gray!20} & \cellcolor{blue!30} \\
        \hline
        \cellcolor{gray!20} & \cellcolor{gray!20} & \cellcolor{gray!20} & \cellcolor{gray!20} & \cellcolor{gray!20} & \cellcolor{gray!20} & \cellcolor{gray!20} & \cellcolor{blue!30} \\
        \hline
        \cellcolor{gray!20} & \cellcolor{gray!20} & \cellcolor{gray!20} & \cellcolor{gray!20} & \cellcolor{gray!20} & \cellcolor{gray!20} & \cellcolor{gray!20} & \cellcolor{blue!30} \\
        \hline
        \cellcolor{gray!20} & \cellcolor{gray!20} & \cellcolor{gray!20} & \cellcolor{gray!20} & \cellcolor{gray!20} & \cellcolor{gray!20} & \cellcolor{gray!20} & \cellcolor{blue!30} \\
        \hline
        \cellcolor{gray!20} & \cellcolor{gray!20} & \cellcolor{gray!20} & \cellcolor{gray!20} & \cellcolor{gray!20} & \cellcolor{gray!20} & \cellcolor{gray!20} & \cellcolor{blue!30} \\
        \hline
        \cellcolor{gray!20} & \cellcolor{gray!20} & \cellcolor{gray!20} & \cellcolor{gray!20} & \cellcolor{gray!20} & \cellcolor{gray!20} & \cellcolor{gray!20} & \cellcolor{blue!30} \\
        \hline
        \cellcolor{blue!30} & \cellcolor{blue!30} & \cellcolor{blue!30} & \cellcolor{blue!30} & \cellcolor{blue!30} & \cellcolor{blue!30} & \cellcolor{blue!30} & \cellcolor{blue!30} \\
        \hline
    \end{tabular}
\end{table}
}



\newcommand{\BasisDCT} {
    \begin{table}[h!]
        \centering
        \caption{Таблица значений $\cos(k\theta)$ для разных $k$ и $\theta$}
        \[
        \begin{array}{c|cccccccc}
        \theta & 0.196 & 0.589 & 0.982 & 1.374 & 1.767 & 2.160 & 2.553 & 2.945 \\
        \hline
        \cos 0\theta & 1. & 1. & 1. & 1. & 1. & 1. & 1. & 1. \\
        \cos 1\theta & 0.981 & 0.831 & 0.556 & 0.195 & -0.195 & -0.556 & -0.831 & -0.981 \\
        \cos 2\theta & 0.924 & 0.383 & -0.383 & -0.924 & -0.924 & -0.383 & 0.383 & 0.924 \\
        \cos 3\theta & 0.831 & -0.195 & -0.981 & -0.556 & 0.556 & 0.981 & 0.195 & -0.831 \\
        \cos 4\theta & 0.707 & -0.707 & -0.707 & 0.707 & 0.707 & -0.707 & -0.707 & 0.707 \\
        \cos 5\theta & 0.556 & -0.981 & 0.195 & 0.831 & -0.831 & -0.195 & 0.981 & -0.556 \\
        \cos 6\theta & 0.383 & -0.924 & 0.924 & -0.383 & -0.383 & 0.924 & -0.924 & 0.383 \\
        \cos 7\theta & 0.195 & -0.556 & 0.831 & -0.981 & 0.981 & -0.831 & 0.556 & -0.195 \\
        \end{array}
        \]
    \end{table}
}


\newcommand{\YQuantize} {
    \begin{center}
        \text{Матрица квантования яркости (Y):}
        \[
        \begin{bmatrix}
        16 & 11 & 10 & 16 & 24 & 40 & 51 & 61 \\
        12 & 12 & 14 & 19 & 26 & 58 & 60 & 55 \\
        14 & 13 & 16 & 24 & 40 & 57 & 69 & 56 \\
        14 & 17 & 22 & 29 & 51 & 87 & 80 & 62 \\
        18 & 22 & 37 & 56 & 68 & 109 & 103 & 77 \\
        24 & 35 & 55 & 64 & 81 & 104 & 113 & 92 \\
        49 & 64 & 78 & 87 & 103 & 121 & 120 & 101 \\
        72 & 92 & 95 & 98 & 112 & 100 & 103 & 99
        \end{bmatrix}
        \]
        \end{center}
}

\newcommand{\CbCrQuantize} {
    \begin{center}
        \text{Матрица квантования хроматических компонент (Cb и Cr):}
        \[
        \begin{bmatrix}
        17 & 18 & 24 & 47 & 99 & 99 & 99 & 99 \\
        18 & 21 & 26 & 66 & 99 & 99 & 99 & 99 \\
        24 & 26 & 56 & 99 & 99 & 99 & 99 & 99 \\
        47 & 66 & 99 & 99 & 99 & 99 & 99 & 99 \\
        99 & 99 & 99 & 99 & 99 & 99 & 99 & 99 \\
        99 & 99 & 99 & 99 & 99 & 99 & 99 & 99 \\
        99 & 99 & 99 & 99 & 99 & 99 & 99 & 99 \\
        99 & 99 & 99 & 99 & 99 & 99 & 99 & 99
        \end{bmatrix}
        \]
        \end{center}
}


\NewEnviron{RLETable}{
\begin{table}[H]
    \centering
    \caption{\text{Последовательность коэффициентов после зигзаг-преобразования}}
    \label{tab:zigzag_seq}
    \renewcommand{\arraystretch}{1.2}
    \begin{tabular}{|c|*{16}{c|}}
    \hline
    \textbf{№} & \textbf{1} & \textbf{2} & \textbf{3} & \textbf{4} & \textbf{5} & \textbf{6} & \textbf{7} & \textbf{8} & \textbf{9} & \textbf{10} & \textbf{11} & \textbf{12} & \textbf{13} & \textbf{14} & \textbf{15} & \textbf{16} \\
    \hline
    \textbf{Значение} & 52 & -3 & 0 & 0 & 0 & 2 & 0 & 0 & 0 & 0 & 0 & 0 & 0 & 0 & 1 & 0 \\
    \hline
    \end{tabular}
\end{table}
}



\NewEnviron{RLEEncodedTable}{
\begin{table}[H]
\centering
\caption{\text{Преобразование последовательности с помощью RLE}}
\label{tab:rle_encoded}
\renewcommand{\arraystretch}{1.2}
\begin{tabular}{|c|c|p{8cm}|}
\hline
\textbf{№} & \textbf{Пара (значение, количество нулей перед)} & \textbf{Пояснение} \\
\hline
1 & (52, 0) & Первое значение, сразу записывается \\
\hline
2 & (-3, 0) & Следующее ненулевое значение без нулей перед ним \\
\hline
3 & (0, 3) & Три последовательных нуля \\
\hline
4 & (2, 0) & Следующий ненулевой коэффициент \\
\hline
5 & (0, 7) & Семь нулей подряд \\
\hline
6 & (1, 0) & Следующий ненулевой коэффициент \\
\hline
7 & (0, 1) & Завершающий ноль \\
\hline
\end{tabular}
\end{table}
}




\newcommand{\TimeDCTlib}{
\begin{table}[ht]
\centering
\caption{Библиотечная реализация DCT)}
\begin{tabular}{|c|c|c|c|c|}
\hline
Время обработки (сек) & Размер до (Кб) & Размер после (Кб) & Ширина & Высота \\
\hline
12.6747 & 1754.59 & 1581.75 & 3500 & 2385 \\
22.3224 & 6860.50 & 980.25  & 5120 & 2880 \\
3.6232  & 2288.86 & 281.88  & 2048 & 1152 \\
5.5511  & 2745.00 & 421.69  & 1440 & 2560 \\
13.9913 & 1745.58 & 1664.48 & 3840 & 2400 \\
21.9425 & 3683.08 & 1055.16 & 5120 & 2880 \\
51.2789 & 10064.11 & 8599.84 & 7680 & 4320 \\
6.1473  & 999.12  & 547.01  & 2560 & 1600 \\
24.3968 & 1258.02 & 1172.49 & 5120 & 3200 \\
12.2057 & 1903.22 & 648.21  & 3840 & 2160 \\
13.8569 & 3414.73 & 977.16  & 3840 & 2400 \\
49.0091 & 6705.57 & 1603.21 & 7680 & 4320 \\
22.0647 & 4566.30 & 1077.42 & 5120 & 2880 \\
41.5755 & 9068.69 & 3607.46 & 6350 & 4285 \\
12.3341 & 1599.10 & 620.09  & 3840 & 2160 \\
21.8722 & 5138.39 & 919.58  & 5120 & 2880 \\
12.1027 & 1187.25 & 632.94  & 3840 & 2160 \\
11.9819 & 3318.69 & 650.59  & 3840 & 2160 \\
11.8780 & 3386.16 & 883.60  & 3840 & 2160 \\
\hline
\end{tabular}
\end{table}
}

\newcommand{\TimeDCTown}{
\begin{table}[ht]
\centering
\caption{Собственная реализация DCT)}
\begin{tabular}{|c|c|c|c|c|}
\hline
Время обработки (сек) & Размер до (Кб) & Размер после (Кб) & Ширина & Высота \\
\hline
6.7837  & 404.31  & 82.92   & 2218 & 1247 \\
80.2578 & 6705.57 & 1603.48 & 7680 & 4320 \\
35.4494 & 4566.30 & 1077.59 & 5120 & 2880 \\
65.5053 & 9068.69 & 3607.05 & 6350 & 4285 \\
19.8254 & 1599.10 & 620.08  & 3840 & 2160 \\
34.7825 & 5138.39 & 919.78  & 5120 & 2880 \\
19.5823 & 662.47  & 219.04  & 3840 & 2160 \\
19.4002 & 1187.25 & 632.96  & 3840 & 2160 \\
19.7136 & 3318.69 & 650.54  & 3840 & 2160 \\
19.6139 & 3386.16 & 883.77  & 3840 & 2160 \\
35.5350 & 6860.50 & 980.02  & 5120 & 2880 \\
4.8934  & 404.31  & 82.90   & 2218 & 1247 \\
\hline
\end{tabular}
\end{table}
}