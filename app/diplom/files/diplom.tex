\documentclass[a4paper,12pt]{extreport}

% --- Кодировки и язык ---
\usepackage[T2A]{fontenc}
\usepackage[utf8]{inputenc}
\usepackage[russian]{babel}

% --- Шрифты ---
\usepackage{pscyr} % Times New Roman
\renewcommand{\rmdefault}{ftm}
\renewcommand{\sfdefault}{ftx}
\renewcommand{\ttdefault}{cmtt}

% --- Поля ---
\usepackage{vmargin}
\setmarginsrb{3cm}{2cm}{1cm}{2cm}{0pt}{0mm}{0pt}{13mm}

% --- Межстрочный интервал, абзац ---
\usepackage{setspace}
\onehalfspacing
\usepackage{indentfirst}
\parindent=1.25cm
\sloppy

% --- Математика ---
\usepackage{amsmath, amssymb, amsfonts, mathtext}

% --- Картинки ---
\usepackage{graphicx}
\graphicspath{{../images/}} % путь к картинкам


% --- Списки ---
\usepackage{enumitem}
\setlist[itemize]{noitemsep, topsep=0pt}
\setlist[enumerate]{noitemsep, topsep=0pt}
\renewcommand{\labelenumi}{\arabic{enumi})}

% --- Таблицы ---
\usepackage{longtable}
\usepackage{floatrow}
\floatsetup[table]{capposition=top}
\usepackage[table]{xcolor}
\usepackage{pifont} % для кружочков

% --- Подписи к рисункам и таблицам ---
\usepackage{caption}
\captionsetup{justification=centering,labelsep=period, figurename=Рисунок, tablename=Таблица}

% --- Оглавление ---
\usepackage{tocloft}
\setcounter{tocdepth}{3}
\setcounter{secnumdepth}{3}
\renewcommand{\cftchapfont}{}
\renewcommand{\cftchappagefont}{}
\renewcommand{\cfttoctitlefont}{\hfil\large\bfseries\MakeUppercase}
\renewcommand{\cftaftertoctitle}{\hfil}

% --- Ссылки и литература ---
\usepackage{url}
\usepackage{cite}
\makeatletter
\renewcommand\@biblabel[1]{#1.}
\makeatother

% --- Команды ---
\newcommand\term[1]{\textit{#1}}
\newcommand\node[1]{\url{#1}}
\newcommand\pic[1]{(см. рисунок~\ref{#1})}
\newcommand\tab[1]{(см. таблицу~\ref{#1})}

% --- Настройки нумерации секций ---
\setcounter{secnumdepth}{1}  % Секции будут нумероваться без дроби
\renewcommand{\thesection}{\arabic{section}}  % Убираем префикс главы
\begin{document}

% --- Титульный лист ---
\begin{titlepage}
    \begin{center}
    
    {\normalsize
    МИНИСТЕРСТВО НАУКИ И ВЫСШЕГО ОБРАЗОВАНИЯ РОССИЙСКОЙ ФЕДЕРАЦИИ\\
    Федеральное государственное автономное образовательное учреждение\\
    высшего образования
    }
    
    \textbf{
    «Национальный исследовательский\\
    Нижегородский государственный университет им. Н.И. Лобачевского»\\
    (ННГУ)
    }
    
    \vfill
    
    \textbf{
    Институт информационных технологий, математики и механики\\
    Кафедра алгебры, геометрии и дискретной математики
    }
    
    \vfill
    
    {\centering
    \normalsize
    Направление подготовки:\\
    «Фундаментальная информатика и информационные технологии»

    Профиль подготовки: «Инженерия программного обеспечения»
    \par}
    
    \vfill
    
    \begin{center}
    \textbf{
    ВЫПУСКНАЯ КВАЛИФИКАЦИОННАЯ РАБОТА\\
    БАКАЛАВРА
    }
    
    \vspace{1.5cm}
    
    \textbf{
    Тема:\\
    «Эволюция алгоритмов сжатия изображений: от RLE до JPEG»
    }
    \end{center}
    
    \vfill
    
    \begin{flushright}
    Выполнил:\\
    студент группы: 3821Б1ФИ2 \\
    Казанцев Евгений Александрович \\
    \hspace{4cm} \rule{4cm}{0.4pt} \\ % место для подписи
    
    \vspace{1cm}
    
    Научный руководитель:\\
    к.ф.-м.н., доцент каф. АГДМ \\
    Смирнова Татьяна Геннадьевна \\
    \hspace{4cm} \rule{4cm}{0.4pt} % место для подписи
    \end{flushright}
    
    \vfill
    
    \begin{center}
    Нижний Новгород \\
    2025
    \end{center}
    
    \end{center}
    \end{titlepage}
    

% --- Таблицы ----
%1
\newcommand{\IMageTable} {
\begin{table}[h!]
    \centering
    \caption{Упрощенная схема цифрового изображения как двумерного массива пикселей}
    \renewcommand{\arraystretch}{1.5}

    \begin{tabular}{|c|c|c|c|}
        \hline
        \cellcolor{gray!20}P(0,0) & \cellcolor{gray!10}P(0,1) & \cellcolor{gray!10}P(0,2) & \cellcolor{gray!20}P(0,3) \\
        \hline
        \cellcolor{gray!10}P(1,0) & \cellcolor{gray!20}P(1,1) & \cellcolor{gray!20}P(1,2) & \cellcolor{gray!10}P(1,3) \\
        \hline
        \cellcolor{gray!20}P(2,0) & \cellcolor{gray!10}P(2,1) & \cellcolor{gray!10}P(2,2) & \cellcolor{gray!20}P(2,3) \\
        \hline
        \cellcolor{gray!10}P(3,0) & \cellcolor{gray!20}P(3,1) & \cellcolor{gray!20}P(3,2) & \cellcolor{gray!10}P(3,3) \\
        \hline
    \end{tabular}
\end{table}
}



%2
\newcommand{\PixelTable} {
\begin{table}[h!]
    \centering
    \renewcommand{\arraystretch}{1.5} % увеличим высоту строк
    \begin{tabular}{|c|c|}
    \hline
    \textbf{RGB значения} & \textbf{Цвет} \\
    \hline
    RGB(255,0,0)   & {\color{red}\ding{108}} \\
    RGB(0,255,0)   & {\color{green}\ding{108}} \\
    RGB(0,0,255)   & {\color{blue}\ding{108}} \\
    \hline
    \end{tabular}
    \caption{Представление пикселей с помощью RGB-значений и их цветов}
\end{table}
}


\newcommand{\AddedBlocks} {
\begin{table}[h!]
    \centering
    \caption{Расширенный блок 8×8 с добавленными пикселами}
    \renewcommand{\arraystretch}{1.5}
    \setlength{\tabcolsep}{0pt} % убираем отступы в ячейках

    \begin{tabular}{|*{8}{p{0.9cm}|}}
        \hline
        \cellcolor{gray!20} & \cellcolor{gray!20} & \cellcolor{gray!20} & \cellcolor{gray!20} & \cellcolor{gray!20} & \cellcolor{gray!20} & \cellcolor{gray!20} & \cellcolor{blue!30} \\
        \hline
        \cellcolor{gray!20} & \cellcolor{gray!20} & \cellcolor{gray!20} & \cellcolor{gray!20} & \cellcolor{gray!20} & \cellcolor{gray!20} & \cellcolor{gray!20} & \cellcolor{blue!30} \\
        \hline
        \cellcolor{gray!20} & \cellcolor{gray!20} & \cellcolor{gray!20} & \cellcolor{gray!20} & \cellcolor{gray!20} & \cellcolor{gray!20} & \cellcolor{gray!20} & \cellcolor{blue!30} \\
        \hline
        \cellcolor{gray!20} & \cellcolor{gray!20} & \cellcolor{gray!20} & \cellcolor{gray!20} & \cellcolor{gray!20} & \cellcolor{gray!20} & \cellcolor{gray!20} & \cellcolor{blue!30} \\
        \hline
        \cellcolor{gray!20} & \cellcolor{gray!20} & \cellcolor{gray!20} & \cellcolor{gray!20} & \cellcolor{gray!20} & \cellcolor{gray!20} & \cellcolor{gray!20} & \cellcolor{blue!30} \\
        \hline
        \cellcolor{gray!20} & \cellcolor{gray!20} & \cellcolor{gray!20} & \cellcolor{gray!20} & \cellcolor{gray!20} & \cellcolor{gray!20} & \cellcolor{gray!20} & \cellcolor{blue!30} \\
        \hline
        \cellcolor{gray!20} & \cellcolor{gray!20} & \cellcolor{gray!20} & \cellcolor{gray!20} & \cellcolor{gray!20} & \cellcolor{gray!20} & \cellcolor{gray!20} & \cellcolor{blue!30} \\
        \hline
        \cellcolor{blue!30} & \cellcolor{blue!30} & \cellcolor{blue!30} & \cellcolor{blue!30} & \cellcolor{blue!30} & \cellcolor{blue!30} & \cellcolor{blue!30} & \cellcolor{blue!30} \\
        \hline
    \end{tabular}
\end{table}
}



\newcommand{\BasisDCT} {
    \begin{table}[h!]
        \centering
        \caption{Таблица значений $\cos(k\theta)$ для разных $k$ и $\theta$}
        \[
        \begin{array}{c|cccccccc}
        \theta & 0.196 & 0.589 & 0.982 & 1.374 & 1.767 & 2.160 & 2.553 & 2.945 \\
        \hline
        \cos 0\theta & 1. & 1. & 1. & 1. & 1. & 1. & 1. & 1. \\
        \cos 1\theta & 0.981 & 0.831 & 0.556 & 0.195 & -0.195 & -0.556 & -0.831 & -0.981 \\
        \cos 2\theta & 0.924 & 0.383 & -0.383 & -0.924 & -0.924 & -0.383 & 0.383 & 0.924 \\
        \cos 3\theta & 0.831 & -0.195 & -0.981 & -0.556 & 0.556 & 0.981 & 0.195 & -0.831 \\
        \cos 4\theta & 0.707 & -0.707 & -0.707 & 0.707 & 0.707 & -0.707 & -0.707 & 0.707 \\
        \cos 5\theta & 0.556 & -0.981 & 0.195 & 0.831 & -0.831 & -0.195 & 0.981 & -0.556 \\
        \cos 6\theta & 0.383 & -0.924 & 0.924 & -0.383 & -0.383 & 0.924 & -0.924 & 0.383 \\
        \cos 7\theta & 0.195 & -0.556 & 0.831 & -0.981 & 0.981 & -0.831 & 0.556 & -0.195 \\
        \end{array}
        \]
    \end{table}
}


\newcommand{\YQuantize} {
    \begin{center}
        \text{Матрица квантования яркости (Y):}
        \[
        \begin{bmatrix}
        16 & 11 & 10 & 16 & 24 & 40 & 51 & 61 \\
        12 & 12 & 14 & 19 & 26 & 58 & 60 & 55 \\
        14 & 13 & 16 & 24 & 40 & 57 & 69 & 56 \\
        14 & 17 & 22 & 29 & 51 & 87 & 80 & 62 \\
        18 & 22 & 37 & 56 & 68 & 109 & 103 & 77 \\
        24 & 35 & 55 & 64 & 81 & 104 & 113 & 92 \\
        49 & 64 & 78 & 87 & 103 & 121 & 120 & 101 \\
        72 & 92 & 95 & 98 & 112 & 100 & 103 & 99
        \end{bmatrix}
        \]
        \end{center}
}

\newcommand{\CbCrQuantize} {
    \begin{center}
        \text{Матрица квантования хроматических компонент (Cb и Cr):}
        \[
        \begin{bmatrix}
        17 & 18 & 24 & 47 & 99 & 99 & 99 & 99 \\
        18 & 21 & 26 & 66 & 99 & 99 & 99 & 99 \\
        24 & 26 & 56 & 99 & 99 & 99 & 99 & 99 \\
        47 & 66 & 99 & 99 & 99 & 99 & 99 & 99 \\
        99 & 99 & 99 & 99 & 99 & 99 & 99 & 99 \\
        99 & 99 & 99 & 99 & 99 & 99 & 99 & 99 \\
        99 & 99 & 99 & 99 & 99 & 99 & 99 & 99 \\
        99 & 99 & 99 & 99 & 99 & 99 & 99 & 99
        \end{bmatrix}
        \]
        \end{center}
}


\NewEnviron{RLETable}{
\begin{table}[H]
    \centering
    \caption{\text{Последовательность коэффициентов после зигзаг-преобразования}}
    \label{tab:zigzag_seq}
    \renewcommand{\arraystretch}{1.2}
    \begin{tabular}{|c|*{16}{c|}}
    \hline
    \textbf{№} & \textbf{1} & \textbf{2} & \textbf{3} & \textbf{4} & \textbf{5} & \textbf{6} & \textbf{7} & \textbf{8} & \textbf{9} & \textbf{10} & \textbf{11} & \textbf{12} & \textbf{13} & \textbf{14} & \textbf{15} & \textbf{16} \\
    \hline
    \textbf{Значение} & 52 & -3 & 0 & 0 & 0 & 2 & 0 & 0 & 0 & 0 & 0 & 0 & 0 & 0 & 1 & 0 \\
    \hline
    \end{tabular}
\end{table}
}



\NewEnviron{RLEEncodedTable}{
\begin{table}[H]
\centering
\caption{\text{Преобразование последовательности с помощью RLE}}
\label{tab:rle_encoded}
\renewcommand{\arraystretch}{1.2}
\begin{tabular}{|c|c|p{8cm}|}
\hline
\textbf{№} & \textbf{Пара (значение, количество нулей перед)} & \textbf{Пояснение} \\
\hline
1 & (52, 0) & Первое значение, сразу записывается \\
\hline
2 & (-3, 0) & Следующее ненулевое значение без нулей перед ним \\
\hline
3 & (0, 3) & Три последовательных нуля \\
\hline
4 & (2, 0) & Следующий ненулевой коэффициент \\
\hline
5 & (0, 7) & Семь нулей подряд \\
\hline
6 & (1, 0) & Следующий ненулевой коэффициент \\
\hline
7 & (0, 1) & Завершающий ноль \\
\hline
\end{tabular}
\end{table}
}




\newcommand{\TimeDCTlib}{
\begin{table}[ht]
\centering
\caption{Библиотечная реализация DCT)}
\begin{tabular}{|c|c|c|c|c|}
\hline
Время обработки (сек) & Размер до (Кб) & Размер после (Кб) & Ширина & Высота \\
\hline
12.6747 & 1754.59 & 1581.75 & 3500 & 2385 \\
22.3224 & 6860.50 & 980.25  & 5120 & 2880 \\
3.6232  & 2288.86 & 281.88  & 2048 & 1152 \\
5.5511  & 2745.00 & 421.69  & 1440 & 2560 \\
13.9913 & 1745.58 & 1664.48 & 3840 & 2400 \\
21.9425 & 3683.08 & 1055.16 & 5120 & 2880 \\
51.2789 & 10064.11 & 8599.84 & 7680 & 4320 \\
6.1473  & 999.12  & 547.01  & 2560 & 1600 \\
24.3968 & 1258.02 & 1172.49 & 5120 & 3200 \\
12.2057 & 1903.22 & 648.21  & 3840 & 2160 \\
13.8569 & 3414.73 & 977.16  & 3840 & 2400 \\
49.0091 & 6705.57 & 1603.21 & 7680 & 4320 \\
22.0647 & 4566.30 & 1077.42 & 5120 & 2880 \\
41.5755 & 9068.69 & 3607.46 & 6350 & 4285 \\
12.3341 & 1599.10 & 620.09  & 3840 & 2160 \\
21.8722 & 5138.39 & 919.58  & 5120 & 2880 \\
12.1027 & 1187.25 & 632.94  & 3840 & 2160 \\
11.9819 & 3318.69 & 650.59  & 3840 & 2160 \\
11.8780 & 3386.16 & 883.60  & 3840 & 2160 \\
\hline
\end{tabular}
\end{table}
}

\newcommand{\TimeDCTown}{
\begin{table}[ht]
\centering
\caption{Собственная реализация DCT)}
\begin{tabular}{|c|c|c|c|c|}
\hline
Время обработки (сек) & Размер до (Кб) & Размер после (Кб) & Ширина & Высота \\
\hline
6.7837  & 404.31  & 82.92   & 2218 & 1247 \\
80.2578 & 6705.57 & 1603.48 & 7680 & 4320 \\
35.4494 & 4566.30 & 1077.59 & 5120 & 2880 \\
65.5053 & 9068.69 & 3607.05 & 6350 & 4285 \\
19.8254 & 1599.10 & 620.08  & 3840 & 2160 \\
34.7825 & 5138.39 & 919.78  & 5120 & 2880 \\
19.5823 & 662.47  & 219.04  & 3840 & 2160 \\
19.4002 & 1187.25 & 632.96  & 3840 & 2160 \\
19.7136 & 3318.69 & 650.54  & 3840 & 2160 \\
19.6139 & 3386.16 & 883.77  & 3840 & 2160 \\
35.5350 & 6860.50 & 980.02  & 5120 & 2880 \\
4.8934  & 404.31  & 82.90   & 2218 & 1247 \\
\hline
\end{tabular}
\end{table}
}

% --- Аннотация ---
\newpage
\chapter*{Аннотация}
Сжатие изображений — применение алгоритмов сжатия данных к изображениям, хранящимся в цифровом виде. 
В результате сжатия уменьшается размер изображения, из-за чего уменьшается время передачи изображения по сети и экономится пространство для хранения.

Сжатие изображений подразделяют на сжатие с потерями качества и сжатие без потерь. 

Преимущество методов сжатия с потерями над методами сжатия без потерь состоит в том, что первые дают возможность большей степени сжатия, при этом, продолжая удовлетворять поставленным требованиям, а именно — искажения должны быть в допустимых пределах чувствительности органов человеческого восприятия.

Методы сжатия с потерями используются для сжатия аналоговых данных — чаще всего звука или изображений.

В таких случаях распакованный файл может сильно отличаться от оригинала по размеру, но практически неотличим для человека «на слух» и «на глаз» в большинстве применений.

Множество методов фокусируются на физических особенностях органов чувств человека. К примеру психоакустические модели определяют то, как сильно звук может быть сжат без ухудшения воспринимаемого человеком качества звука. 

При использовании сжатия с потерями необходимо учитывать, что повторное сжатие обычно приводит к деградации качества. Однако, если повторное сжатие выполняется без каких-либо изменений сжимаемых данных, качество не меняется. Так например, сжатие изображения методом JPEG, восстановление его и повторное сжатие с теми же самыми параметрами не приведет к снижению качества.

\newpage

% --- Оглавление ---
\tableofcontents
\newpage

% --- Основные главы ---
\section*{Введение}
\addcontentsline{toc}{section}{Введение}

Сжатие информации с потерями — это алгоритмическое преобразование данных, направленное на уменьшение их объёма за счёт допустимых потерь качества. 
Необходимость сокращения объёма данных актуальна с появлением цифровых технологий и остается актуальной по сей день. 
Несмотря на развитие технологий хранения и передачи данных, ограничения по объёму памяти и скорости передачи информации по-прежнему требуют эффективных методов сжатия. 
Объёмы данных продолжают расти, их структура усложняется, а качество — повышается. 
Это особенно актуально для мультимедийных данных — изображений, видео и звука.

В качестве основных литературных источников данной работы используются: книга Д. Сэломона «Сжатие данных, изображений и звука» [1], труды П. Пенна [2], а также две книги Т. Рафгардена из серии «Совершенный алгоритм» [3,4]. 
В книге [1] содержится формальное описание алгоритмов сжатия с потерями, в частности алгоритма JPEG, включая его математическую основу — дискретное косинусное преобразование (DCT). 
Труды П. Пенна [2] дают детализированное объяснение реализации алгоритма JPEG на программном уровне. 
Книги [3,4] содержат информацию о базовых алгоритмических принципах, используемых в методах сжатия. 
В ходе работы также использовалась официальная документация библиотеки OpenCV [5] и языка программирования Python 3 [6], а так же библиотеки для создания графического интерфейса PyQt [7].
\\

Вернуться и доработать\\

Существует огромное множество методов сжатия данных, основанных на различных предположениях о структуре сжимаемой информации. 
Все алгоритмы сжатия можно разделить на две основные группы:
Сжатие без потерь — предполагает возможность точного восстановления исходных данных после сжатия.
Сжатие с потерями — допускает потерю части данных ради значительного сокращения объема. 
Этот метод часто применяется к изображениям, видео и аудиоданным, где некоторая потеря качества незаметна для восприятия.

В данной работе рассматривается алгоритм сжатия с потерями на примере широко используемого метода JPEG. 
Алгоритм JPEG основан на использовании дискретного косинусного преобразования (DCT) для преобразования изображения, последующем квантовании и энтропийном кодировании. 
В рамках данной реализации из рассмотрения исключаются этапы энтропийного кодирования, в частности кодирование Хаффмана, а также альтернативные подходы, такие как вейвлет-преобразования.
Этот метод обеспечивает высокую степень сжатия при сохранении приемлемого качества изображения. 
JPEG стал важным этапом в развитии технологий сжатия данных, объединив достижения математических и инженерных исследований своего времени, и дал мощный толчок развитию в области сжатия изображений.

\vspace{1em}

\textbf{Основной задачей} работы является исследование алгоритма JPEG в контексте его математической основы и алгоритмической реализации. 
Предполагается рассмотреть влияние различных параметров сжатия на качество изображения, а также провести сравнение производительности различных реализаций алгоритма. 
В качестве возможных модификаций рассматриваются методы адаптивного квантования и постобработки изображения после декодирования.

\vspace{1em}
\textbf{Целью работы} является изучение развития алгоритмов сжатия с потерями, начиная с фундаментальных математических открытий (например, дискретного косинусного преобразования), простых алгоритмов сжатия без потерь, и заканчивая созданием аналогичного алгоритма JPEG. 
В этой работе реализуется собственный алгоритм, основанный на ключевых этапах стандарта JPEG. 
Его структура почти аналогична базовому варианту. 
В дальнейшем по тексту он будет обозначаться как “модификация алгоритма JPEG” или просто “JPEG”. 
Так же планируется разработать программу с интерфейсом, позволяющую проводить эксперименты по сжатию изображений с использованием модификации алгоритма JPEG.
Программа должна обладать следующими функциями:


\begin{itemize}[label=--]
    \item кодирование изображения методом JPEG с возможностью настройки параметров квантования и качества;
    \item визуализация результатов кодирования и декодирования;
    \item сравнение качества изображения до и после сжатия с использованием метрик SSIM и PSNR;
    \item возможность включения адаптивного квантования для улучшения качества при низких битрейтах.
\end{itemize}

Для достижения поставленной цели сформулированы следующие задачи:
\begin{itemize}[label=--]
    \item изучить теоретическую основу алгоритма JPEG, включая дискретное косинусное преобразование и квантование;
    \item реализовать алгоритм JPEG на языке программирования Python с использованием библиотеки OpenCV;
    \item разработать интерфейс для управления параметрами алгоритма и отображения результатов;
    \item провести эксперименты на различных изображениях, оценить влияние параметров на степень сжатия и качество изображения;
    \item проанализировать результаты экспериментов и сделать выводы о целесообразности использования различных модификаций алгоритма.
\end{itemize}


% --- Основные понятия и определения ---
\newpage
\section{Основные понятия и определения}

Перед тем как углубиться в описание алгоритма JPEG, важно рассмотреть основные понятия, лежащие в основе цифровой обработки и сжатия изображений.

\subsection*{Цифровое изображение}

\textbf{Цифровое изображение} — двумерный массив, где каждая ячейка представляет собой пиксель.

\textbf{Пиксель (pixel)} — наименьшая единица изображения, содержащая информацию о цвете или яркости.

\subsection*{Цветовые модели}

Наиболее распространённой моделью представления цветных изображений является RGB.

\textbf{RGB (Red, Green, Blue)} — аддитивная цветовая модель, в которой каждый цвет формируется путём смешивания трёх базовых компонентов: красного, зелёного и синего. 
Эта модель близка к принципу работы матриц дисплеев и сенсоров цифровых камер, поэтому RGB часто используется как исходный формат цветовых изображений.

В зависимости от глубины цвета, каждый пиксель кодируется определённым числом бит. 
Например, в изображении с глубиной 24 бита (8 бит на каждый из каналов RGB) один пиксель может представлять более 16 миллионов оттенков. 
С увеличением разрешения и глубины цвета возрастает и общий объём изображения, что создаёт необходимость в эффективном сжатии данных.

Однако при сжатии изображений RGB-пространство не является наиболее эффективным. Это связано с тем, что RGB не разделяет яркость и цветовую информацию, а человеческий глаз по-разному чувствителен к этим компонентам. 
В связи с этим перед сжатием изображения, как правило, преобразуются в цветовое пространство \textbf{YCbCr}, где \textbf{Y} отражает яркость (luminance), а \textbf{Cb} и \textbf{Cr} — цветовые отклонения от серого (chrominance). 
Такое разделение позволяет применить хрома-субдискретизацию — уменьшение разрешения цветовых компонент — без заметного ухудшения качества изображения.

\subsection*{Ключевые термины}

Также необходимо упомянуть несколько ключевых понятий, тесно связанных с задачами кодирования изображений:

\textbf{Разрешение изображения} — количество пикселей по горизонтали и вертикали.

\textbf{Избыточность} — повторяющаяся или избыточная информация, которую можно исключить без потери качества или с минимальными потерями.

\textbf{Энтропия} — мера неопределённости или информационной насыщенности данных, определяющая нижнюю границу возможной степени сжатия.

Формат \textbf{JPEG} представляет собой стандарт кодирования изображений с потерями, определяющий последовательность преобразований, квантования и кодирования данных. 
Разработка формата началась в 1986 году с формированием международной рабочей группы под названием \emph{Joint Photographic Experts Group}. 
На тот момент существовало множество несовместимых форматов для хранения графики, что затрудняло обмен изображениями между системами. 
JPEG стал первым по-настоящему универсальным и широко распространённым стандартом сжатия фотоизображений. 
Официальная спецификация была опубликована в 1992 году и используется до сих пор.


% --- Список литературы ---
\begin{thebibliography}{9}

    \bibitem{ref1} Сэломон, Д. Сжатие данных, изображений и звука / Д. Сэломон. – М.: Техносфера, 2006. –
    368 с.



    \bibitem{ref2} Томас, Кормен. Алгоритмы Построение и анализ / Т. Кормен,  Ч. Лейзерсон, Р. Ривест, К, Штайн 
    - 2-е изд., - Москва: Вильямс, 2011. -- 1296 с.


    \bibitem{ref3 } Официальная документация OpenCV. – Режим доступа: -URL: https://docs.opencv.org/4.x/index.html (дата обращения: 19.02.2025). – Текст: электронный.


    \bibitem{ref4} Земцов, А. Н. Сравнительный анализ эффективности методов сжатия изображений на основе дискретного косинусного преобразования и фрактального кодирования (начало) /
    И. В. Петрова, Н. Г. Мамаев. –  Волгоград: Издательство Волгоградского государственного технического университета, 2015. – 8 с.

    \bibitem{ref5} Гарсия, Г. Обработка изображений с помощью OpenCV / Г. Гарсия, О. Суарес, Х. Аранда, Х. Терсеро, И. Грасиа, Н. Энано.
    – Москва: ДМК, 2016. – 210 с.


    
    \bibitem{ref6} Официальная документация PyQt5. – Режим доступа: -URL: https://www.riverbankcomputing.com/ static/Docs/PyQt5/. (дата обращения: 05.01.2025). – Текст: электронный.

    \bibitem{ref6} Официальная документация языка Python. – В режиме доступа: -URL: https://docs.python.org/3/ (дата обращения: 05.01.2025). – Текст: электронный.
    
    \end{thebibliography}

\end{document}