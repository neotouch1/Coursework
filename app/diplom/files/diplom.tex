\documentclass[a4paper,12pt]{extreport}

% --- Кодировки и язык ---
\usepackage[T2A]{fontenc}
\usepackage[utf8]{inputenc}
\usepackage[russian]{babel}

% --- Шрифты ---
\usepackage{pscyr} % Times New Roman
\renewcommand{\rmdefault}{ftm}
\renewcommand{\sfdefault}{ftx}
\renewcommand{\ttdefault}{cmtt}

% --- Поля ---
\usepackage{vmargin}
\setmarginsrb{3cm}{2cm}{1cm}{2cm}{0pt}{0mm}{0pt}{13mm}

% --- Межстрочный интервал, абзац ---
\usepackage{setspace}
\onehalfspacing
\usepackage{indentfirst}
\parindent=1.25cm
\sloppy

% --- Математика ---
\usepackage{amsmath, amssymb, amsfonts, mathtext}

% --- Картинки ---
\usepackage{graphicx}
\graphicspath{{../images/}} % путь к картинкам


% --- Списки ---
\usepackage{enumitem}
\setlist[itemize]{noitemsep, topsep=0pt}
\setlist[enumerate]{noitemsep, topsep=0pt}
\renewcommand{\labelenumi}{\arabic{enumi})}

% --- Таблицы ---
\usepackage{longtable}
\usepackage{floatrow}
\floatsetup[table]{capposition=top}
\usepackage[table]{xcolor}
\usepackage{pifont} % для кружочков

% --- Подписи к рисункам и таблицам ---
\usepackage{caption}
\captionsetup{justification=centering,labelsep=period, figurename=Рисунок, tablename=Таблица}

% --- Оглавление ---
\usepackage{tocloft}
\setcounter{tocdepth}{2}
\setcounter{secnumdepth}{3}
\renewcommand{\cftchapfont}{}
\renewcommand{\cftchappagefont}{}
\renewcommand{\cfttoctitlefont}{\hfil\large\bfseries\MakeUppercase}
\renewcommand{\cftaftertoctitle}{\hfil}

% --- Ссылки и литература ---
\usepackage{url}
\usepackage{cite}
\makeatletter
\renewcommand\@biblabel[1]{#1.}
\makeatother

% --- Команды ---
\newcommand\term[1]{\textit{#1}}
\newcommand\node[1]{\url{#1}}
\newcommand\pic[1]{(см. рисунок~\ref{#1})}
\newcommand\tab[1]{(см. таблицу~\ref{#1})}

% --- Настройки нумерации секций ---
\setcounter{secnumdepth}{2}  % Секции будут нумероваться без дроби
\renewcommand{\thesection}{\arabic{section}}  % Убираем префикс главы
\begin{document}

% --- Титульный лист ---
\begin{titlepage}
    \begin{center}
    
    {\normalsize
    МИНИСТЕРСТВО НАУКИ И ВЫСШЕГО ОБРАЗОВАНИЯ РОССИЙСКОЙ ФЕДЕРАЦИИ\\
    Федеральное государственное автономное образовательное учреждение\\
    высшего образования
    }
    
    \textbf{
    «Национальный исследовательский\\
    Нижегородский государственный университет им. Н.И. Лобачевского»\\
    (ННГУ)
    }
    
    \vfill
    
    \textbf{
    Институт информационных технологий, математики и механики\\
    Кафедра алгебры, геометрии и дискретной математики
    }
    
    \vfill
    
    {\centering
    \normalsize
    Направление подготовки:\\
    «Фундаментальная информатика и информационные технологии»

    Профиль подготовки: «Инженерия программного обеспечения»
    \par}
    
    \vfill
    
    \begin{center}
    \textbf{
    ВЫПУСКНАЯ КВАЛИФИКАЦИОННАЯ РАБОТА\\
    БАКАЛАВРА
    }
    
    \vspace{1.5cm}
    
    \textbf{
    Тема:\\
    «Реализация и исследование алгоритма JPEG для сжатия изображений»
    }
    \end{center}
    
    \vfill
    
    \begin{flushright}
    Выполнил:\\
    студент группы: 3821Б1ФИ2 \\
    Казанцев Евгений Александрович \\
    \hspace{4cm} \rule{4cm}{0.4pt} \\ % место для подписи
    
    \vspace{1cm}
    
    Научный руководитель:\\
    к.ф.-м.н., доцент каф. АГДМ \\
    Смирнова Татьяна Геннадьевна \\
    \hspace{4cm} \rule{4cm}{0.4pt} % место для подписи
    \end{flushright}
    
    \vfill
    
    \begin{center}
    Нижний Новгород \\
    2025
    \end{center}
    
    \end{center}
    \end{titlepage}
    

% --- Таблицы ----
%1
\newcommand{\IMageTable} {
\begin{table}[h!]
    \centering
    \caption{Упрощенная схема цифрового изображения как двумерного массива пикселей}
    \renewcommand{\arraystretch}{1.5}

    \begin{tabular}{|c|c|c|c|}
        \hline
        \cellcolor{gray!20}P(0,0) & \cellcolor{gray!10}P(0,1) & \cellcolor{gray!10}P(0,2) & \cellcolor{gray!20}P(0,3) \\
        \hline
        \cellcolor{gray!10}P(1,0) & \cellcolor{gray!20}P(1,1) & \cellcolor{gray!20}P(1,2) & \cellcolor{gray!10}P(1,3) \\
        \hline
        \cellcolor{gray!20}P(2,0) & \cellcolor{gray!10}P(2,1) & \cellcolor{gray!10}P(2,2) & \cellcolor{gray!20}P(2,3) \\
        \hline
        \cellcolor{gray!10}P(3,0) & \cellcolor{gray!20}P(3,1) & \cellcolor{gray!20}P(3,2) & \cellcolor{gray!10}P(3,3) \\
        \hline
    \end{tabular}
\end{table}
}



%2
\newcommand{\PixelTable} {
\begin{table}[h!]
    \centering
    \renewcommand{\arraystretch}{1.5} % увеличим высоту строк
    \begin{tabular}{|c|c|}
    \hline
    \textbf{RGB значения} & \textbf{Цвет} \\
    \hline
    RGB(255,0,0)   & {\color{red}\ding{108}} \\
    RGB(0,255,0)   & {\color{green}\ding{108}} \\
    RGB(0,0,255)   & {\color{blue}\ding{108}} \\
    \hline
    \end{tabular}
    \caption{Представление пикселей с помощью RGB-значений и их цветов}
\end{table}
}


\newcommand{\AddedBlocks} {
\begin{table}[h!]
    \centering
    \caption{Расширенный блок 8×8 с добавленными пикселами}
    \renewcommand{\arraystretch}{1.5}
    \setlength{\tabcolsep}{0pt} % убираем отступы в ячейках

    \begin{tabular}{|*{8}{p{0.9cm}|}}
        \hline
        \cellcolor{gray!20} & \cellcolor{gray!20} & \cellcolor{gray!20} & \cellcolor{gray!20} & \cellcolor{gray!20} & \cellcolor{gray!20} & \cellcolor{gray!20} & \cellcolor{blue!30} \\
        \hline
        \cellcolor{gray!20} & \cellcolor{gray!20} & \cellcolor{gray!20} & \cellcolor{gray!20} & \cellcolor{gray!20} & \cellcolor{gray!20} & \cellcolor{gray!20} & \cellcolor{blue!30} \\
        \hline
        \cellcolor{gray!20} & \cellcolor{gray!20} & \cellcolor{gray!20} & \cellcolor{gray!20} & \cellcolor{gray!20} & \cellcolor{gray!20} & \cellcolor{gray!20} & \cellcolor{blue!30} \\
        \hline
        \cellcolor{gray!20} & \cellcolor{gray!20} & \cellcolor{gray!20} & \cellcolor{gray!20} & \cellcolor{gray!20} & \cellcolor{gray!20} & \cellcolor{gray!20} & \cellcolor{blue!30} \\
        \hline
        \cellcolor{gray!20} & \cellcolor{gray!20} & \cellcolor{gray!20} & \cellcolor{gray!20} & \cellcolor{gray!20} & \cellcolor{gray!20} & \cellcolor{gray!20} & \cellcolor{blue!30} \\
        \hline
        \cellcolor{gray!20} & \cellcolor{gray!20} & \cellcolor{gray!20} & \cellcolor{gray!20} & \cellcolor{gray!20} & \cellcolor{gray!20} & \cellcolor{gray!20} & \cellcolor{blue!30} \\
        \hline
        \cellcolor{gray!20} & \cellcolor{gray!20} & \cellcolor{gray!20} & \cellcolor{gray!20} & \cellcolor{gray!20} & \cellcolor{gray!20} & \cellcolor{gray!20} & \cellcolor{blue!30} \\
        \hline
        \cellcolor{blue!30} & \cellcolor{blue!30} & \cellcolor{blue!30} & \cellcolor{blue!30} & \cellcolor{blue!30} & \cellcolor{blue!30} & \cellcolor{blue!30} & \cellcolor{blue!30} \\
        \hline
    \end{tabular}
\end{table}
}

% --- Аннотация ---
\newpage
\chapter*{Аннотация}
Сжатие изображений — применение алгоритмов сжатия данных к изображениям, хранящимся в цифровом виде. 
В результате сжатия уменьшается размер изображения, из-за чего уменьшается время передачи изображения по сети и экономится пространство для хранения.

Сжатие изображений подразделяют на сжатие с потерями качества и сжатие без потерь. 

Преимущество методов сжатия с потерями над методами сжатия без потерь состоит в том, что первые дают возможность большей степени сжатия, при этом, продолжая удовлетворять поставленным требованиям, а именно — искажения должны быть в допустимых пределах чувствительности органов человеческого восприятия.

Методы сжатия с потерями используются для сжатия аналоговых данных — чаще всего звука или изображений.

В таких случаях распакованный файл может сильно отличаться от оригинала по размеру, но практически неотличим для человека «на слух» и «на глаз» в большинстве применений.

Множество методов фокусируются на физических особенностях органов чувств человека. К примеру психоакустические модели определяют то, как сильно звук может быть сжат без ухудшения воспринимаемого человеком качества звука. 

При использовании сжатия с потерями необходимо учитывать, что повторное сжатие обычно приводит к деградации качества. Однако, если повторное сжатие выполняется без каких-либо изменений сжимаемых данных, качество не меняется. Так например, сжатие изображения методом JPEG, восстановление его и повторное сжатие с теми же самыми параметрами не приведет к снижению качества.

\newpage

% --- Оглавление ---
\tableofcontents
\newpage

% --- Основные главы ---
\section*{Введение}
\addcontentsline{toc}{section}{Введение}

Сжатие информации с потерями — это алгоритмическое преобразование данных, направленное на уменьшение их объёма за счёт допустимых потерь качества. 
Необходимость сокращения объёма данных актуальна с появлением цифровых технологий и остается актуальной по сей день. 
Несмотря на развитие технологий хранения и передачи данных, ограничения по объёму памяти и скорости передачи информации по-прежнему требуют эффективных методов сжатия. 
Объёмы данных продолжают расти, их структура усложняется, а качество — повышается. 
Это особенно актуально для мультимедийных данных — изображений, видео и звука.

В процессе подготовки и написания работы были использованы как теоретические, 
так и практико-ориентированные источники. 
Основу теоретической части составила книга Д. Сэломона «Сжатие данных, изображений и звука» [1], 
где представлены фундаментальные принципы алгоритмов сжатия, включая алгоритм JPEG 
и его математическую основу — дискретное косинусное преобразование (DCT). 
Для понимания базовых алгоритмических структур и оценки эффективности различных методов были использованы 
материалы из книги «Алгоритмы. Построение и анализ» Т. Кормена и соавторов [2], 
а также статья А. Н. Земцова и коллег, посвящённая сравнительному анализу методов сжатия на основе 
DCT и фрактального кодирования [4].

Практическая реализация алгоритма опиралась на официальную документацию библиотеки OpenCV [3], 
а также на книгу Г. Гарсии и соавторов [5], содержащую конкретные примеры обработки изображений 
с помощью этой библиотеки. При создании графического интерфейса была использована библиотека PyQt5, 
с опорой на официальную документацию [6]. Дополнительные технические сведения, 
связанные с реализацией и настройкой проекта, были получены из официальной документации языка 
программирования Python 3 [7].

Существует огромное множество методов сжатия данных, основанных на различных предположениях о структуре сжимаемой информации. 
Все алгоритмы сжатия можно разделить на две основные группы:
Сжатие без потерь — предполагает возможность точного восстановления исходных данных после сжатия.
Сжатие с потерями — допускает потерю части данных ради значительного сокращения объема. 
Этот метод часто применяется к изображениям, видео и аудиоданным, где некоторая потеря качества незаметна для восприятия.

В данной работе рассматривается алгоритм сжатия с потерями на примере широко используемого метода JPEG. 
Алгоритм JPEG основан на использовании дискретного косинусного преобразования (DCT) для преобразования изображения, последующем квантовании и энтропийном кодировании. 
В рамках данной реализации из рассмотрения исключаются этапы энтропийного кодирования, в частности кодирование Хаффмана, а также альтернативные подходы, такие как вейвлет-преобразования.
Этот метод обеспечивает высокую степень сжатия при сохранении приемлемого качества изображения. 
JPEG стал важным этапом в развитии технологий сжатия данных, объединив достижения математических и инженерных исследований своего времени, и дал мощный толчок развитию в области сжатия изображений.

\vspace{1em}

\textbf{Основной задачей} работы является исследование алгоритма JPEG в контексте его математической основы и алгоритмической реализации. 
Предполагается рассмотреть влияние различных параметров сжатия на качество изображения, а также провести сравнение производительности различных реализаций алгоритма. 
В качестве возможных модификаций рассматриваются методы адаптивного квантования и постобработки изображения после декодирования.

\vspace{1em}
\textbf{Целью работы} является изучение развития алгоритмов сжатия с потерями, начиная с фундаментальных математических открытий (например, дискретного косинусного преобразования), простых алгоритмов сжатия без потерь, и заканчивая созданием аналогичного алгоритма JPEG. 
В этой работе реализуется собственный алгоритм, основанный на ключевых этапах стандарта JPEG. 
Его структура почти аналогична базовому варианту. 
В дальнейшем по тексту он будет обозначаться как “модификация алгоритма JPEG” или просто “JPEG”. 
Так же планируется разработать программу с интерфейсом, позволяющую проводить эксперименты по сжатию изображений с использованием модификации алгоритма JPEG.
Программа должна обладать следующими функциями:


\begin{itemize}[label=--]
    \item кодирование изображения методом JPEG с возможностью настройки параметров квантования и качества;
    \item визуализация результатов кодирования и декодирования;
%     \item сравнение качества изображения до и после сжатия с использованием метрик SSIM и PSNR;'
    % \item возможность включения адаптивного квантования для улучшения качества при низких битрейтах.
\end{itemize}


\textbf{Для достижения поставленной цели сформулированы следующие задачи:}

\begin{itemize}[label=--]
    \item изучить теоретическую основу алгоритма JPEG, включая дискретное косинусное преобразование и квантование;
    \item реализовать алгоритм JPEG на языке программирования Python с использованием библиотеки OpenCV;
    \item разработать интерфейс для управления параметрами алгоритма и отображения результатов используя библиотеку PyQt5;
    \item провести эксперименты на различных изображениях, оценить влияние параметров на степень сжатия и качество изображения;
    \item проанализировать результаты экспериментов и сделать выводы о целесообразности использования различных модификаций алгоритма.
\end{itemize}


% --- Основные понятия и определения ---
\newpage
\section{Основные понятия и определения}

Перед тем как углубиться в описание алгоритма JPEG, важно рассмотреть основные понятия, лежащие в основе цифровой обработки и сжатия изображений.

\subsection*{Цифровое изображение}

\textbf{Цифровое изображение} — двумерный массив, где каждая ячейка представляет собой пиксель.

\textbf{Пиксель (pixel)} — наименьшая единица изображения, содержащая информацию о цвете или яркости.

\subsection*{Цветовые модели}

Наиболее распространённой моделью представления цветных изображений является RGB.

\textbf{RGB (Red, Green, Blue)} — аддитивная цветовая модель, в которой каждый цвет формируется путём смешивания трёх базовых компонентов: красного, зелёного и синего. 
Эта модель близка к принципу работы матриц дисплеев и сенсоров цифровых камер, поэтому RGB часто используется как исходный формат цветовых изображений.

В зависимости от глубины цвета, каждый пиксель кодируется определённым числом бит. 
Например, в изображении с глубиной 24 бита (8 бит на каждый из каналов RGB) один пиксель может представлять более 16 миллионов оттенков. 
С увеличением разрешения и глубины цвета возрастает и общий объём изображения, что создаёт необходимость в эффективном сжатии данных.

Однако при сжатии изображений RGB-пространство не является наиболее эффективным. Это связано с тем, что RGB не разделяет яркость и цветовую информацию, а человеческий глаз по-разному чувствителен к этим компонентам. 
В связи с этим перед сжатием изображения, как правило, преобразуются в цветовое пространство \textbf{YCbCr}, где \textbf{Y} отражает яркость (luminance), а \textbf{Cb} и \textbf{Cr} — цветовые отклонения от серого (chrominance). 
Такое разделение позволяет применить хрома-субдискретизацию — уменьшение разрешения цветовых компонент — без заметного ухудшения качества изображения.

\subsection*{Ключевые термины}

Также необходимо упомянуть несколько ключевых понятий, тесно связанных с задачами кодирования изображений:

\textbf{Разрешение изображения} — количество пикселей по горизонтали и вертикали.

\textbf{Избыточность} — повторяющаяся или избыточная информация, которую можно исключить без потери качества или с минимальными потерями.

\textbf{Энтропия} — мера неопределённости или информационной насыщенности данных, определяющая нижнюю границу возможной степени сжатия.

Формат \textbf{JPEG} представляет собой стандарт кодирования изображений с потерями, определяющий последовательность преобразований, квантования и кодирования данных. 
Разработка формата началась в 1986 году с формированием международной рабочей группы под названием \emph{Joint Photographic Experts Group}. 
На тот момент существовало множество несовместимых форматов для хранения графики, что затрудняло обмен изображениями между системами. 
JPEG стал первым по-настоящему универсальным и широко распространённым стандартом сжатия фотоизображений. 
Официальная спецификация была опубликована в 1992 году и используется до сих пор.

% --- Алгоритмы и стуктуры данных
\section{Алгоритмы и структуры данных}
\subsection{Преобразование цветового пространства (RGB → YCbCr)}

Перед сжатием изображений в формате JPEG зачастую используется преобразование из цветового пространства RGB в YCbCr. 
Это обусловлено как особенностями восприятия цвета человеком, так и требованиями алгоритмов сжатия.


Такое разделение позволяет применить хрома-субдискретизацию — уменьшение разрешения цветовых компонент — без заметного ухудшения качества изображения.
Для RGB с диапазоном значений от 0 до 255, преобразование в YCbCr выполняется по следующим формулам:

$$ Y = 0,299R + 0.587G + 0,114B; $$
$$ Cb = -0,168736R - 0,331264G + 0,5B + 128; $$ 
$$ Cr = 0,5R - 0,418688G - 0,081312B + 128. $$

Здесь значения R, G, B принимаются в диапазоне $[0, 255]$. 
Смещение на 128 единиц в формулах для Cb и Cr необходимо для корректного представления как положительных, так и отрицательных значений в беззнаковом формате (unsigned int).

Поскольку компоненты Cb и Cr менее значимы для восприятия, их можно хранить с пониженным разрешением. 
Наиболее распространённый формат — 4:2:0, при котором на каждые 4 пикселя хранится 4 значения яркости Y и по 1 значению Cb и Cr. 
Это даёт значительное уменьшение объёма данных без существенного ущерба для качества изображения.



\subsection{Преимущества YCbCr}
    \begin{itemize}
        \item Обеспечивает эффективное сжатие без видимых потерь качества;
        \item Разделение яркости и цвета позволяет применять различные методы компрессии к компонентам Y и CbCr;
        \item Поддерживается большинством стандартов обработки изображений и видео.
    \end{itemize}

    Таким образом, преобразование RGB → YCbCr является ключевым шагом в JPEG и других алгоритмах сжатия, 
    позволяющим использовать особенности восприятия цвета человеком для достижения более высокой степени сжатия.




\subsection{Субдискретизация}
На втором этапе сжатия изображения применяется методика, известная как субдискретизация цветности (англ. chroma subsampling). 
Её ключевая идея заключается в сокращении объёма данных, содержащих информацию о цвете, путём уменьшения пространственного разрешения цветовых компонентов. 
Это возможно благодаря физиологической особенности зрительной системы человека: 
она в гораздо большей степени чувствительна к деталям яркости (компонент Y — luminance), чем к изменениям цветовых оттенков (компоненты Cb и Cr — chrominance).

В практической реализации данный метод означает, что цветовая информация (Cr и Cb) кодируется с меньшей точностью по сравнению с яркостной. 
Существует несколько вариантов схем хрома-субдискретизации, которые различаются по степени уменьшения разрешения цветовых компонентов:

4:4:4 — нет сжатия, все компоненты Y, Cb и Cr сохраняют оригинальное разрешение.

4:2:2 — разрешение Cb и Cr уменьшается в два раза по сравнению с Y.

4:2:0 — стандарт для большинства сжатых форматов, где разрешение Cb и Cr уменьшается в два раза как по вертикали, так и по горизонтали.

Схема 4:2:0 является наиболее распространенной для видео и изображений, так как она дает хороший баланс между качеством и сжатием. 
Она предполагает, что на каждые четыре пикселя яркости (расположенных в виде блока 2×2) приходится всего один усреднённый пиксель Cr и один усреднённый пиксель Cb. 
То есть цветовые значения вычисляются как средние значения для группы пикселей, и это значение применяется ко всей группе. 
В результате разрешение цветовых компонентов уменьшается в четыре раза (на 75\%) по сравнению с яркостной компонентой.

Для визуализации можно представить исходное изображение как состоящее из трёх равнозначных слоёв:

Y (яркость),

Cb (синий цветовой компонент),

Cr (красный цветовой компонент).

До субдискретизации каждый из этих компонентов имел одинаковое разрешение и соответственно одинаковый вклад в общий объём данных:

$$
Y + Cb + Cr = 1+1+1=3 \text{ единицы информации}.
$$

После применения схемы 4:2:0, каждая из цветовых компонент уменьшается до $\frac{1}{4}$ исходного размера, а яркостная остаётся без изменений:

$$
Y + \frac{1}{4}Cb + \frac{1}{4}Cr = 1 + 0.25 + 0.25 = 1.5 \text{ единицы информации}.
$$

\begin{figure}[H]
    \centering
    \includegraphics[width=0.7\textwidth]{/home/evgen/Coursework/app/diplom/images/sub_discretization.png}
    \caption{Как происходит преборазование}
    \label{fig:sub_dis}
\end{figure}

Таким образом, достигается двукратное уменьшение общего объёма данных, подлежащих хранению или передаче.

Что важно, такое преобразование происходит практически без визуальных потерь качества. 
Несмотря на то, что цветовые данные редуцируются, человеческий глаз практически не воспринимает различие между оригинальным и сжатым изображением. 
Именно поэтому субдискретизация цветности широко применяется в большинстве алгоритмов сжатия изображений и видео (JPEG, MPEG, H.264),
где критически важно достичь компромисса между качеством и эффективностью хранения данных.



% --- Список литературы ---
\begin{thebibliography}{9}

    \bibitem{ref1} Иванов И.И. Название книги. — М.: Издательство, 2020. — 123 с.
    
    \bibitem{ref2} Петров П.П. Название статьи // Журнал. — 2019. — №4. — С. 45–50.
    
    \end{thebibliography}

\end{document}