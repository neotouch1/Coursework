\section*{Заключение}

В ходе выполнения выпускной квалификационной работы была успешно реализована собственная версия алгоритма 
сжатия изображений, основанная на принципах стандарта JPEG. 
Разработка включала все основные этапы: преобразование цветового пространства, 
блочную обработку, дискретное косинусное преобразование (DCT), квантование и обратное преобразование. 
Вместо кодирования Хаффмана, применяемого в классическом JPEG, 
в данной реализации использовалась алгоритмически более простая схема RLE-сжатия, 
что упростило реализацию и позволило сосредоточиться на исследовании ключевых этапов преобразования изображения.

Был разработан удобный пользовательский интерфейс, позволяющий наглядно управлять некоторыми параметрами 
сжатия (уровень квантования, размер блока) и анализировать результаты в реальном времени. 
Программа визуализирует восстановленное изображение, отображая значения метрик, 
такие как информацию о размере сжатого файла и времени обработки.

Проведены обширные экспериментальные исследования, в том числе:

\begin{itemize}
    \item анализ влияния уровня квантования на качество восстановления изображения;
    \item оценка влияния размера блока на метрики качества и скорость обработки;
    \item сравнение результатов при различных параметрах на одном и том же наборе данных.
\end{itemize}

Эксперименты подтвердили, что параметры алгоритма оказывают значительное влияние 
на компромисс между степенью сжатия, качеством восстановления и производительностью. 
На основе полученных результатов были сделаны выводы, позволяющие адаптировать параметры алгоритма 
под различные практические задачи.

Разработанная реализация и программный интерфейс могут быть полезны как в рамках дальнейших исследований 
в области обработки изображений, так и в образовательных целях, 
благодаря наглядности работы ключевых этапов алгоритма JPEG.

Таким образом, работа соответствует теме «Реализация и исследование алгоритма JPEG для сжатия изображений», 
так как включает как теоретический анализ, так и практическое улучшение и адаптацию классического подхода, 
направленные на повышение гибкости и удобства применения алгоритма в современных условиях.