\section*{Введение}
\addcontentsline{toc}{section}{Введение}

Сжатие информации с потерями — это алгоритмическое преобразование данных, направленное на уменьшение их объёма за счёт допустимых потерь качества. 
Необходимость сокращения объёма данных актуальна с появлением цифровых технологий и остается актуальной по сей день. 
Несмотря на развитие технологий хранения и передачи данных, ограничения по объёму памяти и скорости передачи информации по-прежнему требуют эффективных методов сжатия. 
Объёмы данных продолжают расти, их структура усложняется, а качество — повышается. 
Это особенно актуально для мультимедийных данных — изображений, видео и звука.

В качестве основных литературных источников данной работы используются: книга Д. Сэломона «Сжатие данных, изображений и звука» [1], труды П. Пенна [2], а также две книги Т. Рафгардена из серии «Совершенный алгоритм» [3,4]. 
В книге [1] содержится формальное описание алгоритмов сжатия с потерями, в частности алгоритма JPEG, включая его математическую основу — дискретное косинусное преобразование (DCT). 
Труды П. Пенна [2] дают детализированное объяснение реализации алгоритма JPEG на программном уровне. 
Книги [3,4] содержат информацию о базовых алгоритмических принципах, используемых в методах сжатия. 
В ходе работы также использовалась официальная документация библиотеки OpenCV [5] и языка программирования Python 3 [6], а так же библиотеки для создания графического интерфейса PyQt [7].
\\

Вернуться и доработать\\

Существует огромное множество методов сжатия данных, основанных на различных предположениях о структуре сжимаемой информации. 
Все алгоритмы сжатия можно разделить на две основные группы:
Сжатие без потерь — предполагает возможность точного восстановления исходных данных после сжатия.
Сжатие с потерями — допускает потерю части данных ради значительного сокращения объема. 
Этот метод часто применяется к изображениям, видео и аудиоданным, где некоторая потеря качества незаметна для восприятия.

В данной работе рассматривается алгоритм сжатия с потерями на примере широко используемого метода JPEG. 
Алгоритм JPEG основан на использовании дискретного косинусного преобразования (DCT) для преобразования изображения, последующем квантовании и энтропийном кодировании. 
В рамках данной реализации из рассмотрения исключаются этапы энтропийного кодирования, в частности кодирование Хаффмана, а также альтернативные подходы, такие как вейвлет-преобразования.
Этот метод обеспечивает высокую степень сжатия при сохранении приемлемого качества изображения. 
JPEG стал важным этапом в развитии технологий сжатия данных, объединив достижения математических и инженерных исследований своего времени, и дал мощный толчок развитию в области сжатия изображений.

\vspace{1em}

\textbf{Основной задачей} работы является исследование алгоритма JPEG в контексте его математической основы и алгоритмической реализации. 
Предполагается рассмотреть влияние различных параметров сжатия на качество изображения, а также провести сравнение производительности различных реализаций алгоритма. 
В качестве возможных модификаций рассматриваются методы адаптивного квантования и постобработки изображения после декодирования.

\vspace{1em}
\textbf{Целью работы} является изучение развития алгоритмов сжатия с потерями, начиная с фундаментальных математических открытий (например, дискретного косинусного преобразования), простых алгоритмов сжатия без потерь, и заканчивая созданием аналогичного алгоритма JPEG. 
В этой работе реализуется собственный алгоритм, основанный на ключевых этапах стандарта JPEG. 
Его структура почти аналогична базовому варианту. 
В дальнейшем по тексту он будет обозначаться как “модификация алгоритма JPEG” или просто “JPEG”. 
Так же планируется разработать программу с интерфейсом, позволяющую проводить эксперименты по сжатию изображений с использованием модификации алгоритма JPEG.
Программа должна обладать следующими функциями:


\begin{itemize}[label=--]
    \item кодирование изображения методом JPEG с возможностью настройки параметров квантования и качества;
    \item визуализация результатов кодирования и декодирования;
    \item сравнение качества изображения до и после сжатия с использованием метрик SSIM и PSNR;
    \item возможность включения адаптивного квантования для улучшения качества при низких битрейтах.
\end{itemize}

Для достижения поставленной цели сформулированы следующие задачи:
\begin{itemize}[label=--]
    \item изучить теоретическую основу алгоритма JPEG, включая дискретное косинусное преобразование и квантование;
    \item реализовать алгоритм JPEG на языке программирования Python с использованием библиотеки OpenCV;
    \item разработать интерфейс для управления параметрами алгоритма и отображения результатов;
    \item провести эксперименты на различных изображениях, оценить влияние параметров на степень сжатия и качество изображения;
    \item проанализировать результаты экспериментов и сделать выводы о целесообразности использования различных модификаций алгоритма.
\end{itemize}
