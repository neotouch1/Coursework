\section*{Введение}
\addcontentsline{toc}{section}{Введение}

Сжатие информации с потерями — это алгоритмическое преобразование данных, направленное на уменьшение их объёма за счёт допустимых потерь качества. 
Необходимость сокращения объёма данных актуальна с появлением цифровых технологий и остается актуальной по сей день. 
Несмотря на развитие технологий хранения и передачи данных, ограничения по объёму памяти и скорости передачи информации по-прежнему требуют эффективных методов сжатия. 
Объёмы данных продолжают расти, их структура усложняется, а качество — повышается. 
Это особенно актуально для мультимедийных данных — изображений, видео и звука.

В процессе подготовки и написания работы были использованы как теоретические, 
так и практико-ориентированные источники. 
Основу теоретической части составила книга Д. Сэломона «Сжатие данных, изображений и звука» [1], 
где представлены фундаментальные принципы алгоритмов сжатия, включая алгоритм JPEG 
и его математическую основу — дискретное косинусное преобразование (DCT). 
Для понимания базовых алгоритмических структур и оценки эффективности различных методов были использованы 
материалы из книги «Алгоритмы. Построение и анализ» Т. Кормена и соавторов [2], 
а также статья А. Н. Земцова и коллег, посвящённая сравнительному анализу методов сжатия на основе 
DCT и фрактального кодирования [4].

Практическая реализация алгоритма опиралась на официальную документацию библиотеки OpenCV [3], 
а также на книгу Г. Гарсии и соавторов [5], содержащую конкретные примеры обработки изображений 
с помощью этой библиотеки. При создании графического интерфейса была использована библиотека PyQt5, 
с опорой на официальную документацию [6]. Дополнительные технические сведения, 
связанные с реализацией и настройкой проекта, были получены из официальной документации языка 
программирования Python 3 [7].

Существует огромное множество методов сжатия данных, основанных на различных предположениях о структуре сжимаемой информации. 
Все алгоритмы сжатия можно разделить на две основные группы:
Сжатие без потерь — предполагает возможность точного восстановления исходных данных после сжатия.
Сжатие с потерями — допускает потерю части данных ради значительного сокращения объема. 
Этот метод часто применяется к изображениям, видео и аудиоданным, где некоторая потеря качества незаметна для восприятия.

В данной работе рассматривается алгоритм сжатия с потерями на примере широко используемого метода JPEG. 
Алгоритм JPEG основан на использовании дискретного косинусного преобразования (DCT) для преобразования изображения, последующем квантовании и энтропийном кодировании. 
В рамках данной реализации из рассмотрения исключаются этапы энтропийного кодирования, в частности кодирование Хаффмана, а также альтернативные подходы, такие как вейвлет-преобразования.
Этот метод обеспечивает высокую степень сжатия при сохранении приемлемого качества изображения. 
JPEG стал важным этапом в развитии технологий сжатия данных, объединив достижения математических и инженерных исследований своего времени, и дал мощный толчок развитию в области сжатия изображений.

\vspace{1em}

\textbf{Основной задачей} работы является исследование алгоритма JPEG в контексте его математической основы и алгоритмической реализации. 
Предполагается рассмотреть влияние различных параметров сжатия на качество изображения, а также провести сравнение производительности различных реализаций алгоритма. 
В качестве возможных модификаций рассматриваются методы адаптивного квантования и постобработки изображения после декодирования.

\vspace{1em}
\textbf{Целью работы} является изучение развития алгоритмов сжатия с потерями, начиная с фундаментальных математических открытий (например, дискретного косинусного преобразования), простых алгоритмов сжатия без потерь, и заканчивая созданием аналогичного алгоритма JPEG. 
В этой работе реализуется собственный алгоритм, основанный на ключевых этапах стандарта JPEG. 
Его структура почти аналогична базовому варианту. 
В дальнейшем по тексту он будет обозначаться как “модификация алгоритма JPEG” или просто “JPEG”. 
Так же планируется разработать программу с интерфейсом, позволяющую проводить эксперименты по сжатию изображений с использованием модификации алгоритма JPEG.
Программа должна обладать следующими функциями:


\begin{itemize}[label=--]
    \item кодирование изображения методом JPEG с возможностью настройки параметров квантования и качества;
    \item визуализация результатов кодирования и декодирования;
%     \item сравнение качества изображения до и после сжатия с использованием метрик SSIM и PSNR;'
    % \item возможность включения адаптивного квантования для улучшения качества при низких битрейтах.
\end{itemize}


\textbf{Для достижения поставленной цели сформулированы следующие задачи:}

\begin{itemize}[label=--]
    \item изучить теоретическую основу алгоритма JPEG, включая дискретное косинусное преобразование и квантование;
    \item реализовать алгоритм JPEG на языке программирования Python с использованием библиотек OpenCV, numpy, scipy;
    \item разработать интерфейс для управления параметрами алгоритма и отображения результатов, используя библиотеку PyQt5;
    \item провести эксперименты на различных изображениях, оценить влияние параметров на степень сжатия и качество изображения;
    \item проанализировать результаты экспериментов и сделать выводы о целесообразности использования различных модификаций алгоритма.
\end{itemize}
