\section{Алгоритмы и структуры данных}
\subsection{Преобразование цветового пространства (RGB → YCbCr)}

Перед сжатием изображений в формате JPEG зачастую используется преобразование из цветового пространства RGB в YCbCr. 
Это обусловлено как особенностями восприятия цвета человеком, так и требованиями алгоритмов сжатия.


Такое разделение позволяет применить хрома-субдискретизацию — уменьшение разрешения цветовых компонент — без заметного ухудшения качества изображения.
Для RGB с диапазоном значений от 0 до 255, преобразование в YCbCr выполняется по следующим формулам:

$$ Y = 0,299R + 0.587G + 0,114B; $$
$$ Cb = -0,168736R - 0,331264G + 0,5B + 128; $$ 
$$ Cr = 0,5R - 0,418688G - 0,081312B + 128. $$

Здесь значения R, G, B принимаются в диапазоне $[0, 255]$. 
Смещение на 128 единиц в формулах для Cb и Cr необходимо для корректного представления как положительных, так и отрицательных значений в беззнаковом формате (unsigned int).

Поскольку компоненты Cb и Cr менее значимы для восприятия, их можно хранить с пониженным разрешением. 
Наиболее распространённый формат — 4:2:0, при котором на каждые 4 пикселя хранится 4 значения яркости Y и по 1 значению Cb и Cr. 
Это даёт значительное уменьшение объёма данных без существенного ущерба для качества изображения.


\subsection{Преимущества YCbCr}