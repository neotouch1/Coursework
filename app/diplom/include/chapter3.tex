\section{Руководство пользователя}

\subsection{Назначение программы}

Разработанная программа реализует алгоритм сжатия изображений, повторяющий основные этапы стандарта JPEG. 
В отличие от оригинального алгоритма, на этапе энтропийного кодирования используется метод RLE (Run-Length Encoding) 
вместо кодирования по Хаффману. 
Это позволяет упростить реализацию, сохранив при этом все основные свойства алгоритма JPEG.

Программа предназначена для демонстрации полного цикла кодирования и декодирования изображения:

\begin{itemize}
    \item переход от $\textbf{RGB}$ к $\textbf{YCbCr}$
    \item разбиение на блоки $8 \times 8$
    \item дискретное косинус-преобразование (DCT)
    \item квантование
    \item последовательное сканирование (зигзаг)
    \item кодирование RLE
    \item обратные преобразования (IDCT и т. д.)
    \item восстановление RGB-изображения

\end{itemize}



\clearpage
%%%%%%%%%%%%%%%%%%%%%%%%%%%%%%%%%%%%%%%%%%%%%%%%
\subsection{Интерфейс программы}

На рисунке ~\ref{fig:interface} представлено главное окно приложения.

\begin{figure}[h!]
    \centering
    \includegraphics[width=0.7\textwidth]{/home/evgen/Coursework/app/diplom/images/interface.png}
    \caption{Главное окно приложения.}
    \label{fig:interface}
\end{figure}



\subsubsection{Основные элементы интерфейса:}

\begin{itemize}
    \item \textbf{Открыть изображение}

    Позволяет выбрать изображение в формате PNG или JPEG для обработки.
    
    \begin{figure}[h!]
        \centering
        \includegraphics[width=0.5\textwidth]{/home/evgen/Coursework/app/diplom/images/win_choice.png}
        \caption{Окно выбора изображений.}
        \label{fig:win_choice}
    \end{figure}



    \item \textbf{Выбрать реализацию DCT}

    В графическом интерфейсе приложения предусмотрена возможность выбора реализации дискретного косинусного преобразования (DCT). 
    Пользователю предлагаются два варианта:

    Собственная реализация DCT, разработанная вручную в рамках дипломного проекта;
    
    Готовая реализация из библиотеки SciPy (scipy.fftpack.dct), используемая в качестве эталонной и производительной альтернативы.
    
    Это позволяет провести наглядное сравнение корректности и эффективности обеих реализаций, а также удостовериться в соответствии собственной реализации общепринятым стандартам.

    \begin{figure}[h!]
        \centering
        \includegraphics[width=0.8\textwidth]{/home/evgen/Coursework/app/diplom/images/choice_dct.png}
        \caption{Выбрать реализацию DCT.}
        \label{fig:choice_dct}
    \end{figure}



    \item \textbf{Задать уровень сжатия}
    
    Предусмотрена возможность установки уровня сжатия, который напрямую влияет на коэффициенты квантования. 
    Более высокий уровень сжатия приводит к агрессивному округлению высокочастотных коэффициентов, 
    что уменьшает объём выходного файла, но может снизить качество восстановленного изображения. 
    Напротив, меньшие значения сохраняют больше информации, обеспечивая лучшее качество при увеличенном размере файла.

    Это позволяет пользователю находить баланс между качеством изображения и степенью компрессии в зависимости от задач.
    \begin{figure}[h!]
        \centering
        \includegraphics[width=0.9\textwidth]{/home/evgen/Coursework/app/diplom/images/lvl_compres.png}
        \caption{Выбрать уровень сжатия.}
        \label{fig:lvl_compres}
    \end{figure}

    \FloatBarrier

    \item \textbf{Обработать изображение}
    
    После выбора параметров пользователь может запустить процесс обработки, 
    нажав на кнопку \textbf{«Обработать изображение»}. 
    Программа выполняет полный цикл сжатия и восстановления изображения, повторяя основные этапы алгоритма JPEG.

    Сначала изображение преобразуется из цветового пространства RGB в YCbCr.
    Затем изображение разбивается на блоки.
    К каждому блоку применяется дискретное косинусное преобразование (DCT), 
    переводя данные из пространственной области в частотную. 
    Полученные коэффициенты проходят этап квантования, 
    при котором незначительные высокочастотные компоненты подавляются — это и есть основное сжатие с потерями.

    После квантования коэффициенты сканируются по зигзагообразной траектории, что группирует нули, 
    появившиеся в результате округлений. 
    Затем выполняется кодирование методом RLE.

    На финальном этапе происходит декодирование, и вывод получившегося изображения в в окне приложения.

    \begin{figure}[h!]
        \centering
        \includegraphics[width=0.9\textwidth]{/home/evgen/Coursework/app/diplom/images/continue.png}
        \caption{Дополнительные элементы.}
        \label{fig:continue}
    \end{figure}


    После завершения обработки становятся доступными дополнительные элементы интерфейса, 
    позволяющие более подробно проанализировать результат.

    \FloatBarrier

    \item \textbf{Сохранить изображение}
    
    Позволяет сохранить на диске восстановленное после сжатия изображение в формате JPEG.

    \begin{figure}[h!]
        \centering
        \includegraphics[width=0.9\textwidth]{/home/evgen/Coursework/app/diplom/images/save_img.png}
        \caption{Сохранить изображение.}
        \label{fig:save_img}
    \end{figure}


    \FloatBarrier
    \item \textbf{Показать карту энергии DCT-коэффициентов}
    
    Данный элемент визуализирует вклад различных частотных компонент в изображение до этапа квантования. 
    Это позволяет понять, какие частоты наиболее значимы в передаче визуальной информации


    \begin{figure}[h!]
        \centering
        \includegraphics[width=0.9\textwidth]{/home/evgen/Coursework/app/diplom/images/map_energy.png}
        \caption{Карта энергии DCT-коэффициентов.}
        \label{fig:map_energy}
    \end{figure}


    \FloatBarrier

    \item \textbf{Показать карту энергии квантованных блоков} 

    Отображает распределение энергии после квантования. 
    Данная визуализация помогает оценить потери, вызванные округлением коэффициентов.

    \begin{figure}[h!]
        \centering
        \includegraphics[width=0.9\textwidth]{/home/evgen/Coursework/app/diplom/images/map_quant.png}
        \caption{Карта энергии квантованных блоков.}
        \label{fig:map_quant}
    \end{figure}

\end{itemize}

\FloatBarrier
В нижней части окна приложения отображаются числовые параметры, 
предоставляющие пользователю информацию о производительности и эффективности алгоритма сжатия:
\begin{itemize}
    \item \textbf{Размер изображений} — отображает объём изображения до и после обработки, в килобайтах. 
    Этот позволяет пользователю сравнить исходный размер файла с результатом сжатия и оценить, 
    насколько сильно уменьшился объём данных.

    \item \textbf{Время обработки} — показывает общее время, затраченное на выполнение всех этапов алгоритма, 
    включая преобразование цветового пространства, разбиение на блоки, дискретное косинусное преобразование, 
    квантование, кодирование с помощью RLE и последующее восстановление изображения. 
    Этот показатель даёт представление о производительности реализации и позволяет сравнивать 
    скорость работы разных вариантов DCT или уровней сжатия.
\end{itemize}


\begin{figure}[h!]
    \centering
    \includegraphics[width=0.9\textwidth]{/home/evgen/Coursework/app/diplom/images/time_size.png}
    \caption{Числовые параметры.}
    \label{fig:time_size}
\end{figure}
