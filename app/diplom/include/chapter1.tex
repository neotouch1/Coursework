\newpage
\section{Основные понятия и определения}

Перед тем как углубиться в описание алгоритма JPEG, важно рассмотреть основные понятия, лежащие в основе цифровой обработки и сжатия изображений.

\subsection*{Цифровое изображение}

\textbf{Цифровое изображение} — двумерный массив, где каждая ячейка представляет собой пиксель.

\textbf{Пиксель (pixel)} — наименьшая единица изображения, содержащая информацию о цвете или яркости.

\subsection*{Цветовые модели}

Наиболее распространённой моделью представления цветных изображений является RGB.

\textbf{RGB (Red, Green, Blue)} — аддитивная цветовая модель, в которой каждый цвет формируется путём смешивания трёх базовых компонентов: красного, зелёного и синего. 
Эта модель близка к принципу работы матриц дисплеев и сенсоров цифровых камер, поэтому RGB часто используется как исходный формат цветовых изображений.

В зависимости от глубины цвета, каждый пиксель кодируется определённым числом бит. 
Например, в изображении с глубиной 24 бита (8 бит на каждый из каналов RGB) один пиксель может представлять более 16 миллионов оттенков. 
С увеличением разрешения и глубины цвета возрастает и общий объём изображения, что создаёт необходимость в эффективном сжатии данных.

Однако при сжатии изображений RGB-пространство не является наиболее эффективным. Это связано с тем, что RGB не разделяет яркость и цветовую информацию, а человеческий глаз по-разному чувствителен к этим компонентам. 
В связи с этим перед сжатием изображения, как правило, преобразуются в цветовое пространство \textbf{YCbCr}, где \textbf{Y} отражает яркость (luminance), а \textbf{Cb} и \textbf{Cr} — цветовые отклонения от серого (chrominance). 
Такое разделение позволяет применить хрома-субдискретизацию — уменьшение разрешения цветовых компонент — без заметного ухудшения качества изображения.

\subsection*{Ключевые термины}

Также необходимо упомянуть несколько ключевых понятий, тесно связанных с задачами кодирования изображений:

\textbf{Разрешение изображения} — количество пикселей по горизонтали и вертикали.

\textbf{Избыточность} — повторяющаяся или избыточная информация, которую можно исключить без потери качества или с минимальными потерями.

\textbf{Энтропия} — мера неопределённости или информационной насыщенности данных, определяющая нижнюю границу возможной степени сжатия.

Формат \textbf{JPEG} представляет собой стандарт кодирования изображений с потерями, определяющий последовательность преобразований, квантования и кодирования данных. 
Разработка формата началась в 1986 году с формированием международной рабочей группы под названием \emph{Joint Photographic Experts Group}. 
На тот момент существовало множество несовместимых форматов для хранения графики, что затрудняло обмен изображениями между системами. 
JPEG стал первым по-настоящему универсальным и широко распространённым стандартом сжатия фотоизображений. 
Официальная спецификация была опубликована в 1992 году и используется до сих пор.