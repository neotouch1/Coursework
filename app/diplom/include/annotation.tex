\newpage
\chapter*{Аннотация}
Сжатие изображений — применение алгоритмов сжатия данных к изображениям, хранящимся в цифровом виде. 
В результате сжатия уменьшается размер изображения, из-за чего уменьшается время передачи изображения по сети и экономится пространство для хранения.

Сжатие изображений подразделяют на сжатие с потерями качества и сжатие без потерь. 

Преимущество методов сжатия с потерями над методами сжатия без потерь состоит в том, что первые дают возможность большей степени сжатия, при этом, продолжая удовлетворять поставленным требованиям, а именно — искажения должны быть в допустимых пределах чувствительности органов человеческого восприятия.

Методы сжатия с потерями используются для сжатия аналоговых данных — чаще всего звука или изображений.

В таких случаях распакованный файл может сильно отличаться от оригинала по размеру, но практически неотличим для человека «на слух» и «на глаз» в большинстве применений.

Множество методов фокусируются на физических особенностях органов чувств человека. К примеру психоакустические модели определяют то, как сильно звук может быть сжат без ухудшения воспринимаемого человеком качества звука. 

При использовании сжатия с потерями необходимо учитывать, что повторное сжатие обычно приводит к деградации качества. Однако, если повторное сжатие выполняется без каких-либо изменений сжимаемых данных, качество не меняется. Так например, сжатие изображения методом JPEG, восстановление его и повторное сжатие с теми же самыми параметрами не приведет к снижению качества.

\newpage